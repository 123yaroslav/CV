%-----------------------------------------------------------------------------------------------------------------------------------------------%
%	The MIT License (MIT)
%
%	Copyright (c) 2021 Jitin Nair
%
%	Permission is hereby granted, free of charge, to any person obtaining a copy
%	of this software and associated documentation files (the "Software"), to deal
%	in the Software without restriction, including without limitation the rights
%	to use, copy, modify, merge, publish, distribute, sublicense, and/or sell
%	copies of the Software, and to permit persons to whom the Software is
%	furnished to do so, subject to the following conditions:
%	
%	THE SOFTWARE IS PROVIDED "AS IS", WITHOUT WARRANTY OF ANY KIND, EXPRESS OR
%	IMPLIED, INCLUDING BUT NOT LIMITED TO THE WARRANTIES OF MERCHANTABILITY,
%	FITNESS FOR A PARTICULAR PURPOSE AND NONINFRINGEMENT. IN NO EVENT SHALL THE
%	AUTHORS OR COPYRIGHT HOLDERS BE LIABLE FOR ANY CLAIM, DAMAGES OR OTHER
%	LIABILITY, WHETHER IN AN ACTION OF CONTRACT, TORT OR OTHERWISE, ARISING FROM,
%	OUT OF OR IN CONNECTION WITH THE SOFTWARE OR THE USE OR OTHER DEALINGS IN
%	THE SOFTWARE.
%	
%
%-----------------------------------------------------------------------------------------------------------------------------------------------%

%----------------------------------------------------------------------------------------
%	DOCUMENT DEFINITION
%----------------------------------------------------------------------------------------

% article class because we want to fully customize the page and not use a cv template
\documentclass[a4paper,12pt]{article}

%----------------------------------------------------------------------------------------
%	FONT
%----------------------------------------------------------------------------------------

% % fontspec allows you to use TTF/OTF fonts directly
% \usepackage{fontspec}
% \defaultfontfeatures{Ligatures=TeX}

% % modified for ShareLaTeX use
% \setmainfont[
% SmallCapsFont = Fontin-SmallCaps.otf,
% BoldFont = Fontin-Bold.otf,
% ItalicFont = Fontin-Italic.otf
% ]
% {Fontin.otf}

%----------------------------------------------------------------------------------------
%	PACKAGES
%----------------------------------------------------------------------------------------
\usepackage{url}
\usepackage{parskip} 	

% Cyrillic support
\usepackage[T2A]{fontenc}
\usepackage[utf8]{inputenc}
\usepackage[russian]{babel}

%other packages for formatting
\RequirePackage{color}
\RequirePackage{graphicx}
\usepackage[usenames,dvipsnames]{xcolor}
\usepackage[scale=0.9]{geometry}

%tabularx environment
\usepackage{tabularx}

%for lists within experience section
\usepackage{enumitem}

% centered version of 'X' col. type
\newcolumntype{C}{>{\centering\arraybackslash}X} 

%to prevent spillover of tabular into next pages
\usepackage{supertabular}
\usepackage{tabularx}
\newlength{\fullcollw}
\setlength{\fullcollw}{0.47\textwidth}

%custom \section
\usepackage{titlesec}				
\usepackage{multicol}
\usepackage{multirow}

%CV Sections inspired by: 
%http://stefano.italians.nl/archives/26
\titleformat{\section}{\Large\scshape\raggedright}{}{0em}{}[\titlerule]
\titlespacing{\section}{0pt}{10pt}{10pt}

%for publications
\usepackage[style=authoryear,sorting=ynt, maxbibnames=2]{biblatex}
\usepackage{csquotes}

%Setup hyperref package, and colours for links
\usepackage[unicode, draft=false]{hyperref}
\definecolor{linkcolour}{rgb}{0,0.2,0.6}
\hypersetup{colorlinks,breaklinks,urlcolor=linkcolour,linkcolor=linkcolour}
\addbibresource{citations.bib}
\setlength\bibitemsep{1em}

%for social icons
\usepackage{fontawesome5}

%debug page outer frames
%\usepackage{showframe}


% job listing environments
\newenvironment{jobshort}[2]
    {
    \begin{tabularx}{\linewidth}{@{}l X r@{}}
    \textbf{#1} & \hfill &  #2 \\[3.75pt]
    \end{tabularx}
    }
    {
    }

\newenvironment{joblong}[2]
    {
    \begin{tabularx}{\linewidth}{@{}l X r@{}}
    \textbf{#1} & \hfill &  #2 \\[3.75pt]
    \end{tabularx}
    \begin{minipage}[t]{\linewidth}
    \begin{itemize}[nosep,after=\strut, leftmargin=1em, itemsep=3pt,label=--]
    }
    {
    \end{itemize}
    \end{minipage}    
    }



%----------------------------------------------------------------------------------------
%	BEGIN DOCUMENT
%----------------------------------------------------------------------------------------
\begin{document}

% non-numbered pages
\pagestyle{empty} 

%----------------------------------------------------------------------------------------
%	TITLE
%----------------------------------------------------------------------------------------

% \begin{tabularx}{\linewidth}{ @{}X X@{} }
% \huge{Your Name}\vspace{2pt} & \hfill \emoji{incoming-envelope} email@email.com \\
% \raisebox{-0.05\height}\faGithub\ username \ | \
% \raisebox{-0.00\height}\faLinkedin\ username \ | \ \raisebox{-0.05\height}\faGlobe \ mysite.com  & \hfill \emoji{calling} number
% \end{tabularx}

\begin{tabularx}{\linewidth}{@{} C @{}}
\Huge{Рогоза Ярослав Евгеньевич} \\[7.5pt]
\href{tel:+79227542689}{\raisebox{-0.05\height}\faMobile \ +7 (922) 754-26-89} \ \textbar\ \ 
\href{mailto:r.yaroslav1w@gmail.com}{\raisebox{-0.05\height}\faEnvelope \ r.yaroslav1w@gmail.com} \ \textbar\ \ 
Telegram: @ne_onnn \\
Москва, Россия
\end{tabularx}

%----------------------------------------------------------------------------------------
% EXPERIENCE SECTIONS
%----------------------------------------------------------------------------------------

%Интересы / Ключевые слова / Резюме
\section{О себе}
Выпускник Экономического факультета МГУ имени М.В. Ломоносова. Увлекаюсь статистикой и машинным обучением. 
Работаю ведущим продуктовым аналитиком, провожу полный цикл A/B-тестирования, развиваю data-driven подход, 
исследую причинно-следственные связи и оцениваю экономический эффект гипотез. Быстро обучаюсь, аккуратно 
отношусь к срокам и задачам, умею работать в команде и самостоятельно решать проблемы.

%Experience
\section{Опыт работы}

\begin{joblong}{Ведущий продуктовый аналитик — ВкусВилл}{Июль 2023 — настоящее время}
\item Полный цикл A/B-тестирования, участие в построении общего древа метрик
\item Развитие data-driven подхода, исследования причинно-следственных связей (CausalImpact, pymc, causalpy)
\item Использование ML-моделей для прогнозирования и анализа (ETNA, Prophet)
\item Разработка автоматической отчётности и телеграм-ботов с оповещениями (Power BI, Superset, aiogram)
\item ETL-процессы на базе Apache Airflow, работа с DWH (Greenplum, ClickHouse)
\item Онбординг аналитиков, участие в собеседованиях и подготовке тестовых задач
\end{joblong}

\begin{jobshort}{Ключевые результаты}{ } 
Помог обнаружить и исправить аномалии в вознаграждении курьеров (экономия $\approx 5\%$ ФОТ). 
Инициировал тест автоматической маршрутизации курьеров, что снизило click-to-eat. 
\newline Суммарный охват пользователей отчётов — $\approx 100$ человек в месяц, телеграм-ботов — $\approx 500$ сотрудников поддержки.
\end{jobshort}
  
%Проекты (при необходимости)
%\section{Проекты}
%
%\begin{tabularx}{\linewidth}{ @{}l r@{} }
%\textbf{Название проекта} & \hfill \href{https://link}{Ссылка} \\[3.75pt]
%\multicolumn{2}{@{}X@{}}{Краткое описание проекта и вклада.}  \\
%\end{tabularx}

%----------------------------------------------------------------------------------------
%	EDUCATION
%----------------------------------------------------------------------------------------
\section{Образование}
\begin{tabularx}{\linewidth}{@{}l X@{}}	
2025 & Бакалавр, Экономика — \textbf{МГУ им. М.В. Ломоносова}, Экономический факультет, Москва \\
\end{tabularx}

\section{Курсы}
\begin{tabularx}{\linewidth}{@{}l X@{}}
2024 & Введение в глубокое обучение — Институт теоретической и математической физики МГУ \\
2024 & ML-инжиниринг — ИТМО AI talent hub \texttimes{} Karpov.courses \\
2024 & Practical Bayes (pymc) — Экономический факультет МГУ \\
2023 & Аналитик данных — Яндекс Практикум \\
2023 & Машинное обучение для решения прикладных задач — ИТМФ МГУ \\
\end{tabularx}

%----------------------------------------------------------------------------------------
%	PUBLICATIONS
%----------------------------------------------------------------------------------------
\section{Публикации}
\begin{refsection}[citations.bib]
\nocite{*}
\printbibliography[heading=none]
\end{refsection}

%----------------------------------------------------------------------------------------
%	SKILLS
%----------------------------------------------------------------------------------------
\section{Навыки}
\begin{tabularx}{\linewidth}{@{}l X@{}}
Аналитика &  \normalsize{A/B-тестирование, причинно-следственный анализ (CausalImpact, pymc, causalpy), EDA, Ad-hoc анализ, эконометрика}\\
Data &  \normalsize{SQL, ClickHouse, Greenplum, Apache Airflow (ETL)}\\
Python &  \normalsize{pandas, numpy, scikit-learn, LightGBM, CatBoost, XGBoost, ETNA, Prophet, PyTorch, NLTK, folium}\\
BI / Визуализация &  \normalsize{Power BI, Apache Superset}\\
Инструменты &  \normalsize{aiogram, ambrosia}\\
\end{tabularx}

\section{Языки}
\begin{tabularx}{\linewidth}{@{}l X@{}}
Русский & Родной \\
Английский & C1 — Продвинутый \\
\end{tabularx}

\vfill
\center{\footnotesize Last updated: \today}

\end{document}
